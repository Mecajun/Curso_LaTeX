%%% AULA 2
\begin{frame}{Itens, enumerados e descrições}
\begin{LaTeXcode}[Exemplo de itens com marcador]
\LCmdArg{begin}{itemize} \n
\LCmd{item} Primeiro item; \n
\LCmdArg{begin}{itemize} \n
\LCmd{item} Sub-item; \n
\LCmd{item Outro} sub-item; \n
\LCmdArg{end}{itemize} \n
\LCmd{item} Último item. \n
\LCmdArg{end}{itemize}
\end{LaTeXcode}

Produz:
\begin{LaTeXoutput}
\begin{itemize}
\item Primeiro item;
\begin{itemize}
\item Sub-item;
\item Outro sub-item;
\end{itemize}
\item Último item.
\end{itemize}
\end{LaTeXoutput}
\end{frame}

\begin{frame}{Itens, enumerados e descrições}
\begin{LaTeXcode}[Exemplo com numeração]
\LCmdArg{begin}{enumerate} \n
\LCmd{item} Primeiro; \n
\LCmd{item} Segundo; \n
\LCmdArg{begin}{enumerate} \n
\LCmd{item} Sub-item; \n
\LCmd{item} Sub-item. \n
\LCmdArg{end}{enumerate} \n
\LCmdArg{end}{enumerate}
\end{LaTeXcode}

Produz:
\begin{LaTeXoutput}
\begin{enumerate}
\item Primeiro;
\item Segundo;
\begin{enumerate}
\item Sub-item;
\item Sub-item.
\end{enumerate}
\end{enumerate}
\end{LaTeXoutput}
\end{frame}

\begin{frame}{Itens, enumerados e descrições}
\begin{LaTeXcode}[Exemplo de descrição]
\LCmdArg{begin}{description} \n
\LCmd{item} [Windows] Espécie de virus de computador
(costuma ser notado ao gerar a mensagem
`{}`Falha Geral de Proteção'{}'); \n
\LCmd{item} [MacOS] Sistema operacional da Apple; \n
\LCmd{item} [Linux] Sistema operacional livre. \n
\LCmdArg{end}{description}
\end{LaTeXcode}

Produz:

\begin{LaTeXoutput}
\begin{description}
\item [Windows] Sistema operacional da Microsoft;
\item [MacOS] Sistema operacional da Apple;
\item [Linux] Sistema operacional livre.
\end{description}
\end{LaTeXoutput}
\end{frame}

\begin{frame}{Construido Tabelas}
O ambiente \Lenv{tabular} é usado para definir tabelas em modo texto (que não contenham nenhuma ou pouca matemática).

\begin{LaTeXcode}[Sintaxe]
\LCmdArg{begin}{tabular}\Larg{colunas}
linhas \\
\LCmdArg{end}{tabular}

\LCmdArg{begin}{tabular*}\Larg{tamanho}\LOptArg[posição]{colunas}
linhas \\
\LCmdArg{end}{tabular*}
\end{LaTeXcode}
\end{frame}

\begin{frame}{Ambiente \Lenv{tabular}}
\begin{description}
	\item [pos] Posicionamento vertical em relação ao texto (Detalhado melhor aqui)
	\item [tamanho] Este argumento se aplica apenas para o ambiente %\verb+tabular*+
	\item [colunas] Comando de formatação das colunas. Aonde é definido a posição do texto em cada coluna bem como as bordas laterais e espaçamentos.
\end{description}
	\begin{description}
	\item [l] Conteúdo da coluna alinhado a esquerda
	\item [c] Conteúdo da coluna alinhado ao centro
	\item [r] Conteúdo da coluna alinhado a direita
	\item [|] Desenha uma linha vertical
	\item [||] Desenha duas linhas verticais , uma seguida da outra
	\end{description}
\end{frame}

\begin{frame}{Ambiente \Lenv{tabular}}
\begin{description}
	\item [p\{wd\}] O texto na coluna é inserido em linha com largura \emph{wd} e a primeira linha é alinhada com as outras colunas.
	\item [@\{texto\}] Insere em cada linha o texto ou expressão
	\item [Linhas] Cada linha deve terminar com \string\ \string\. Dentro da linha as celulas de cada coluna são separadas por \& conforme da definido antes.
	\item [\string\hline] Este comando desenha um traço horizontal depois da linha da coluna anterior e antes da subsequente.
\end{description}
\end{frame}

\begin{frame}{Ambiente \Lenv{tabular}}
\begin{LaTeXcode}[Exemplo]
\LCmdArg{begin}{tabular}\Larg{l|c|r}  
\LCmd{hline}\n
Elemento   \&  Porcentagem \& Fator \string\\ \n
\string\hline\string\hline \n
Ferro      \&  10          \& 3     \string\\ \string\hline \n
Cloro      \&  33          \& 7     \string\\ \string\hline \n
Oxigênio   \&  51          \& 1     \string\\ \string\hline \n
\LCmdArg{end}{tabular}
\end{LaTeXcode}

\begin{block}{Observação}
As letras ``l'', ``c'' e ``r'' referem-se ao posicionamento do conteúdo nas colunas da tabela.
\end{block}
\end{frame}

\begin{frame}{Ambiente \Lenv{tabular}}
\fontsize{10}{11}\selectfont
Produz:
\begin{LaTeXoutput}
\begin{tabular}{l|c|r}                   
\hline
Elemento    &  Porcentagem  & Fator \\ \hline \hline
Ferro   &  10   & 3     \\ \hline
Cloro   &  33   & 7     \\ \hline
Oxigênio  &  51   & 1 \\ \hline
\end{tabular}
\end{LaTeXoutput}
\end{frame}

\begin{frame}{Ambiente \Lenv{tabular}}
\begin{itemize}
\item \texttt{@\{\}} na especificação do comando tabular resulta em uma divisão com espaçamento zero. Podemos usar para alinhar números pelo ponto decimal;
\item \LCmd{multicolumn} serve para juntar colunas da tabela.
\end{itemize}
\end{frame}

\begin{frame}{Ambiente \Lenv{tabular}}
\begin{LaTeXcode}[Exemplo]
\LCmdArg{begin}{tabular}\Larg{c r @\{,\} l}\n
Expressão \& \LCmd{multicolumn}\Larg{2}\Larg{c}\Larg{Valor} \string\\ \string\hline\n
\$\string\pi\$ \& 3 \& 1415 \string\\ \n
\$\string\pi\string^2\$ \& 9 \& 869 \string\\ \n
\$\string\pi\string^3\$ \& 31 \& 0062 \n
\LCmdArg{end}{tabular}
\end{LaTeXcode}

Produz:
\begin{LaTeXoutput}
\begin{tabular}{c r @{,} l}
Expressão & \multicolumn{2}{c}{Valor} \\ \hline
$\pi$ & 3 & 1415 \\
$\pi^2$ & 9 & 869 \\
$\pi^3$ & 31 & 0062
\end{tabular}
\end{LaTeXoutput}
\end{frame}

\begin{frame}{Referências cruzadas}
\begin{block}{Referenciando seções, subseções, fórmulas, etc.}
\begin{itemize}
\item Para marcar: \LCmdArg{label}{marca};
\item Para referenciar: \LCmdArg{ref}{marca};
\item Para referenciar trocando o nome do link: \LCmdOptArg{hyperref}{marca}{texto}
\item Referenciando a página: \LCmdArg{pageref}{marca}.
\end{itemize}
\end{block}

\begin{block}{Observação}
As referências são armazenadas no arquivo .AUX e por isto pode ser necessária mais de uma compilação para resolver as pendências.
\end{block}
\end{frame}

\begin{frame}{Referências cruzadas}

\begin{LaTeXcode}[Exemplo]
\LCmdArg{begin}{equation} \LCmdArg{label}{eqn:integral}\n
\string\int\ x\string\,\string\mathrm\{d\}x\n
\LCmdArg{end}{equation}\n
A equação~(\LCmdArg{ref}{eqn:integral}) define \string\dots
\end{LaTeXcode}

Produz:
\begin{LaTeXoutput}
\begin{equation} \label{eqn:integral}
\int x\,\mathrm{d}x
\end{equation}
A equação~(\ref{eqn:integral}) define \dots
\end{LaTeXoutput}
\end{frame}

\begin{frame}{Citações e versos}
\begin{LaTeXcode}[Exemplo]
Exclamou Alice enquanto avançava com cuidado
pelo bosque:\n
\LCmdArg{begin}{quote}\n
Foi o chá mais idiota de que participei em
toda a minha vida!\n
\LCmdArg{end}{quote}
\end{LaTeXcode}

Produz:

\begin{LaTeXoutput}
Exclamou Alice enquanto avançava com cuidado
pelo bosque:
\begin{quote}\normalfont
Foi o chá mais idiota de que participei em
toda a minha vida!
\end{quote}
\end{LaTeXoutput}
\end{frame}

\begin{frame}{Versos}
\begin{LaTeXcode}[Exemplo de versos]
Esta é uma poesia sem sentido retirada de
`{}`Alice Através do Espelho'{}':
\nn
\LCmdArg{begin}{center}\n
\LCmdArg{textbf}{Pargarávio}\n
\LCmdArg{end}{center}\n
\LCmdArg{begin}{verse}\n
Solumbrava, e os lubriciosos touvos \string\\ \n
Em vertigiros persondavam as verdentes; \string\\ \n
Trisciturnos calavam-se os gaiolouvos \string\\ \n
E os porverdidos estriguilavam fientes.\n
\LCmdArg{end}{verse}
\end{LaTeXcode}
\end{frame}

\begin{frame}{Versos}
Produz:

\begin{LaTeXoutput}
Esta é uma poesia sem sentido retirada de
``Alice Através do Espelho'':

\begin{center}
\textbf{Pargarávio}
\end{center}
\begin{verse}
Solumbrava, e os lubriciosos touvos \\
Em vertigiros persondavam as verdentes; \\
Trisciturnos calavam-se os gaiolouvos \\
E os porverdidos estriguilavam fientes.
\end{verse}
\end{LaTeXoutput}
\end{frame}

\begin{frame}{Figuras e tabelas}
São \emph{corpos flutuantes}. Obtidos usando-se os ambientes:
\begin{LaTeXcode}[Figuras e Tabelas]
\LCmdArg{begin}{figure}\Lopt[especificação] \n
... \n
\LCmdArg{caption}{texto} \n
\LCmdArg{end}{figure} \n
e \n
\LCmdArg{begin}{table}\Lopt[especificação] \n
... \n
\LCmdArg{caption}{texto} \n
\LCmdArg{end}{table}
\end{LaTeXcode}

\begin{block}{Observação}
\LCmdArg{caption}{\dots} serve para incluir uma legenda.
\end{block}
\end{frame}

\begin{frame}{Figuras e tabelas}
A especificação pode ser um ou mais dos seguintes (não será necessariamente seguido pelo \LaTeX):
\begin{description}
\item [h] aqui;
\item [t] alto da página;
\item [b] embaixo da página;
\item [p] página especial;
\item [!] não considera alguns parâmetros internos.
\end{description}

\begin{block}{}
A ordem em que são usados é relevante -- maior prioridade é dada ao primeiro e menor ao último.
\end{block}
\end{frame}

\begin{frame}{Figuras e tabelas}
\fontsize{10}{11}\selectfont

\begin{LaTeXcode}[Exemplo]
\LCmdArg{begin}{table}\Lopt[!tp] \n
\LCmdArg{caption}{Tabela sem sentido} 
\LCmdArg{label}{tab:semsentido} \n
\LCmd{centering} \n
\LCmdArg{begin}{tabular}\Larg{l|l} \string\hline \n
Parâmetro \& Valor \string\\ \string\hline\string\hline \n
XYZ \& 123 \string\\ \n
ABC \& 321 \string\\ \string\hline \n
\LCmdArg{end}{tabular} \n
\LCmdArg{end}{table} \n
A Tabela\string~\LCmdArg{ref}{tab:semsentido} apresenta \string\dots
\end{LaTeXcode}

\begin{block}{Observações}
\begin{itemize}
\item \texttt{\string\centering} serve para centralizar o tabular;
\item comando \texttt{\string\caption\{\dots\}} usado acima do tabular devido a ABNT;
\item comando \texttt{\string\label\{\dots\}} deve ser usado após o comando \texttt{\string\caption\{\dots\}}.
\end{itemize}
\end{block}
\end{frame}

\begin{frame}{Figuras e tabelas}
\fontsize{10}{11}\selectfont

Produz:

\begin{LaTeXoutput}
\begin{table}
\caption{Tabela sem sentido}% 
\label{tab:semsentido}
\centering
\begin{tabular}{l|l} \hline
Parâmetro & Valor \\ \hline\hline
XYZ & 123 \\
ABC & 321 \\ \hline
\end{tabular}
\end{table}
A Tabela~\ref{tab:semsentido} apresenta \ldots
\end{LaTeXoutput}
\end{frame}

\begin{frame}{Modos do \TeX}
\begin{description}
\item [Modo parágrafo] Divide texto em linhas, parágrafos e páginas; é o modo normal do \TeX;
\item [Modo LR] Descarrega os tipos sem dividir texto; obtido usando-se \LCmdArg{mbox}{} (\LCmd{mbox} pode ser usado quando não desejamos que uma palavra seja dividida em duas linhas/páginas, por exemplo, \LCmdArg{mbox}{555-1234});
\item [Modo matemático] Para produzir fórmulas matemáticas; Obtido usando-se \texttt{\bs(\dots\bs)}, \texttt{\$\dots\$}, \LCmdArg{begin}{displaymath}\dots\LCmdArg{end}{displaymath}, \texttt{\bs\ls\dots\bs\rs}, \LCmdArg{begin}{equation}\dots \LCmdArg{end}{equation} e \LCmdArg{begin}{eqnarray}\dots \LCmdArg{end}{eqnarray}.
\end{description}
\end{frame}


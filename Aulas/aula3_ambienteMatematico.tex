%% Aula 3 - Matemática 
%% Versão com Macros

\begin{frame}{Comandos \LCmd{newcommand} e \LCmd{newtheorem}}
\begin{itemize}
\item O comando \LCmd{newcommand} é usado para definir novos comandos (macros);
\item Sua sintaxe é:
\begin{LaTeXcode}[\LCmd{newcommand}]
\LNCOA{newcommand}{cmd}[args]{definição}\n
ou\n
\LNCOA{newcommand}{cmd}{definição}
\end{LaTeXcode}
\item No primeiro argumento fica o nome do novo comando, o argumento opcional é o número de argumentos do novo comando (numerados a partir de 1) e referenciados com ``\#'' na definição;
\end{itemize}
\end{frame}

\begin{frame}{\LCmd{newcommand}}
\begin{LaTeXcode}[Exemplo]
\LNCOA{newcommand}{titulo}[1]{\lb\string\Large \string\textbf\{\#1\}\rb}\n

\LCmdArg{titulo}{Meu título} 
\end{LaTeXcode}
Produz:
\begin{LaTeXoutput}
\titulo{Meu título}
\end{LaTeXoutput}
\end{frame}

\begin{frame}{\LCmd{newtheorem}}
O comando \LCmd{newtheorem} permite definir teoremas, definições, exemplos, etc.

\begin{LaTeXcode}[Exemplo]
\string\newtheorem\{exe\}\{Exemplo\}\n
...\n
\LCmdArg{begin}{exe}\n
Este é um exemplo.\n
\LCmdArg{end}{exe}
\end{LaTeXcode}
Produz:
\begin{LaTeXoutput}
\textbf{Exemplo 1} \textit{Este é um exemplo.}
\end{LaTeXoutput}
\end{frame}

\begin{frame}{Comando \LCmd{newenvironment}}
O comando \LCmd{newenvironment} permite criar novos ambientes, permitindo personalizar uma região aonde terão comandos executados antes e depois.
\LNCOA{newenvironment}{nomeAmbiente}[numArgumentos]{Comandos Antes}\Larg{Comandos Depois}

\end{frame}

\begin{frame}{Comando \LCmd{newenvironment}}

\begin{LaTeXcode}[Exemplo]
\LCmdArg{newenvironment}{minhaTabela}
\{ \% Comandos executados Antes
\LCmdArg{begin}{table}\n
\LCmd{centering}\n
\LCmdArg{begin}{tabular}\Larg{c r @\{,\} l}\n
Expressão \& \LCmd{multicolumn}\Larg{2}\Larg{c}\Larg{Valor} \string\\ \string\hline\n
\}\n
\{\% Comandos executados depois\n
\LCmdArg{end}{tabular}\n
\LCmdArg{end}{table}\n
\}
\end{LaTeXcode}

\end{frame}

\begin{frame}{Comando \LCmd{newenvironment}}
\begin{LaTeXcode}[Uso do novo Ambiente]
  \LCmdArg{begin}{minhaTabela}\n
\$\string\pi\$ \& 3 \& 1415 \string\\ \n
\$\string\pi\string^2\$ \& 9 \& 869 \string\\ \n
\$\string\pi\string^3\$ \& 31 \& 0062 \n
  \LCmdArg{end}{minhaTabela}
\end{LaTeXcode}
\end{frame}

\begin{frame}{Produzindo textos com matemática}
\begin{itemize}
\item \texttt{\$\dots\$} para produzir fórmulas dentro de um parágrafo em linha com o texto;
\item \texttt{\bs\ls\dots\bs\rs} para produzir equações destacadas do parágrafo;
\item \LCmdArg{begin}{equation}\dots \LCmdArg{label}{marca}\LCmdArg{end}{equation} para produzir uma equação numerada e destacada do parágrafo e poder referencia-la usando \LCmdArg{ref}{marca}.
\end{itemize}
\end{frame}

\begin{frame}{Exemplos}
\begin{LaTeXcode}[Exemplo 1]
Tome \$x\$ e adicione \$y\$. Você obterá \$x+y\$.
Outra equação importante é a 
do segundo grau \bs\ls ax\string^2+bx+c=0\bs\rs\ cuja solução é dada pela \string\emph\{Fórmula de Bhaskara\}.
\nn
Seja, por exemplo, a equação\string~(\LCmdArg{ref}{eqn:exemplo}).\n
\LCmdArg{begin}{equation}\n
2x\string^2-3x+1=0\n
\LCmdArg{label}{eqn:exemplo}\n
\LCmdArg{end}{equation}\n
Podemos dizer que \$x=1\$ é uma
solução da equação.
\end{LaTeXcode}

\end{frame}

\begin{frame}{Exemplo 1}


Produz:
\begin{LaTeXoutput}
Tome $x$ e adicione $y$. Você obterá $x+y$.
Outra equação importante é a 
do segundo grau \[ax^2+bx+c=0\] cuja solução é dada pela \emph{Fórmula de Bhaskara}.

Seja, por exemplo, a Equação~(\ref{eqn:exemplo}):
\begin{equation}\label{eqn:exemplo}
2x^2-3x+1=0
\end{equation}
Podemos dizer que $x=1$ é uma
solução da equação.
\end{LaTeXoutput}

\end{frame}

\begin{frame}{Exemplo 2}


\begin{LaTeXcode}[Exemplo 2]
\LCmd{TeX}\LCmd{\textvisiblespace} deve ser pronunciado como\n
\$\string\tau\string\epsilon\string\chi\$.
\end{LaTeXcode}
Produz:
\begin{LaTeXoutput}
\TeX\ deve ser pronunciado como
$\tau\epsilon\chi$.
\end{LaTeXoutput}

\end{frame}

\begin{frame}{Subscritos e expoentes}
\begin{center}\let\tt\ttfamily
\begin{tabular}{cc}
$x^{2}$ & \tt\$x\string^\{2\}\$ \\
$x^{y^{2}}$ & \tt\$x\string^\{y\string^\{2\}\}\$ \\
$x_{1}^{2}$ & \tt\$x\string_\{1\}\string^\{2\}\$\\
\end{tabular}
\end{center}
\end{frame}

\begin{frame}{Frações}
\begin{LaTeXcode}
\bs\ls\ a/b \bs\rs
\end{LaTeXcode}
Produz: 
\begin{block}{}
\[a/b\]
\end{block}
\begin{LaTeXcode}
\bs\ls \string\frac\{a\}\{b\} \bs\rs
\end{LaTeXcode}
Produz:
\begin{block}{}
\[\frac{a}{b}\]
\end{block}
\end{frame}

\begin{frame}{Frações}
\begin{itemize}
\item \texttt{/} é preferível quando existe pouca coisa na fração e o espaço é pequeno;
\begin{LaTeXcode}[Exemplo]
\$2\string^\{1/2\}\$ e \$2\string^\string\frac\{1\}\{x+1\}\$.
\end{LaTeXcode}

Produz:
\begin{block}{}
\[2^{1/2} \text{\quad e\quad} 2^\frac{1}{x+1}\]
\end{block}

\medskip

\item Muitas vezes \texttt{\string\frac} parece ruim quando usado dentro de um parágrafo com \texttt{\$\dots\$};
\item Como pode ser visto a partir do exemplo, mesmo nos exponentes o comando \LCmd{frac} não produz um resultado agradável.
\end{itemize}
\end{frame}

\begin{frame}{Integral}
\begin{LaTeXcode}[Exemplo de integral dupla]
\texttt{\string\iint\string\sin\ x\string\cos\ y\string\,\string\mathrm\{d\}x\string\,\string\mathrm\{d\}y}
\end{LaTeXcode}
Produz:
\begin{LaTeXoutput}
\[\iint \sin x\cos y\,\mathrm{d}x\,\mathrm{d}y\]
\end{LaTeXoutput}

\begin{block}{Observações}
\begin{itemize}
\item \texttt{\string\iint} produz $\displaystyle\iint$ 
e \texttt{\string\int\string\int} produz $\displaystyle\int\int$;
\item Comandos \LCmd{iint}, \LCmd{iiint}, \LCmd{iiiint} e \LCmd{idotsint} são produzidos pelo pacote \Lsty{amsmath}. Sem esses comandos é necessário tratar os espaços entre as várias partes da integral.
\end{itemize}
\end{block}
\end{frame}

\begin{frame}{Somatório}
\begin{LaTeXcode}[Exemplo de somatório]
\texttt{\string\sum\string_\{i=1\}\string^\string\infty\ a\string_i}
\end{LaTeXcode}

\bigskip

Produz:
\begin{block}{}
\[\sum_{i=1}^\infty a_i\]
\end{block}
\end{frame}

\begin{frame}{Integral definida}
\begin{LaTeXcode}[Usando limites de integração]
\string\int\string_0\string^\string\frac\{1\}\{2\} x\string^2\string\,\string\mathrm\{d\}x
\end{LaTeXcode}
Produz:
\begin{LaTeXoutput}
\[\int_0^\frac{1}{2} x^2\mathrm{d}x\]
\end{LaTeXoutput}
\end{frame}

\begin{frame}{Matemática em linha ou destacada do parágrafo}
Diferenças na aparência usando \texttt{\$\dots\$} ou \texttt{\bs\ls\dots\bs\rs}; Contraste:
\begin{LaTeXcode}[Matemática em linha]
\$\LCmd{lim}\_\Larg{n\LCmd{to}\LCmd{infty}}\LCmd{sum}\_\Larg{i=1}\^{}n 1/i\$
\end{LaTeXcode}
que produz:
\begin{block}{}
$\lim_{n\to\infty}\sum_{i=1}^n1/i$
\end{block}
com:
\begin{LaTeXcode}[Matemática em display (destacando)]
\bs\ls\ \LCmd{lim}\_\Larg{n\LCmd{to}\LCmd{infty}}\LCmd{sum}\_\Larg{i=1}\^{}n
\LCmd{frac}\Larg{1}\Larg{i}\ \bs\rs
\end{LaTeXcode}
que produz:
\begin{block}{}
\[\lim_{n\to\infty}\sum_{i=1}^n\frac{1}{i}\]
\end{block}
\end{frame}

\begin{frame}{Mais um exemplo}
\begin{LaTeXcode}[Usando quantificador, conjuntos e desigualdade]
\LCmd{forall} x\LCmd{in}\LCmdArg{mathbb}{R}:x\^{}2\LCmd{geq} 0
\end{LaTeXcode}

Produz:
\begin{LaTeXoutput}
\[\forall x\in\mathbb{R}:x^2\geq 0\]
\end{LaTeXoutput}
\end{frame}

\begin{frame}{Igualdades e desigualdades}
\begin{block}{Igualdades e desigualdades}
\begin{center}\let\tt\ttfamily
\begin{tabular}{*6c}
\tt= & \LCmd{neq} & \tt> & \tt< & \LCmd{leq} & \LCmd{geq} \\
$=$ & $\neq$ & $>$ & $<$ & $\leq$ & $\geq$
\end{tabular}
\end{center}
\end{block}

\end{frame}

\begin{frame}{Acentos em modo matemático}
\begin{block}{Acentos em modo matemático}
\begin{center}\tabcolsep=4pt
\begin{tabular}{*4c}
\LCmdArg{hat}{a} & \LCmdArg{grave}{a} & \LCmdArg{bar}{a} & \LCmdArg{check}{a}  \\
$\hat{a}$ & $\grave{a}$ & $\bar{a}$ & $\check{a}$ \\
\LCmdArg{dot}{a} & \LCmdArg{vec}{a} & \LCmdArg{breve}{a} & \LCmdArg{widetilde}{abc} \\
$\dot{a}$ & $\vec{a}$ & $\breve{a}$ & $\widetilde{abc}$ \\
\LCmdArg{tilde}{a} & \LCmdArg{ddot}{a} & \LCmdArg{widehat}{abc} & \LCmdArg{acute}{a} \\
$\tilde{a}$ & $\ddot{a}$ & $\widehat{abc}$ & $\acute{a}$
\end{tabular}
\end{center}
\end{block}
\end{frame}

\begin{frame}{Fontes do modo matemático}
\begin{block}{Alguns fontes do modo matemático}
\begin{center}
\begin{tabular}{*3l}
Caligráfico & \LCmdArg{mathcal}{C} & $\mathcal{C}$ \\
Redobrado	& \LCmdArg{mathbb}{R} & $\mathbb{R}$ \\
Bold 		& \LCmdArg{mathbf}{B} & $\mathbf{B}$ \\
Roman 		& \LCmdArg{mathrm}{M} & $\mathrm{M}$
\end{tabular}
\end{center}
\end{block}

\begin{block}{Observações}
Para poder ser usado, o fonte \texttt{\string\mathbb\{\dots\}} necessita o pacote \Lsty{amssymb}.
\end{block}
\end{frame}

\begin{frame}{Espaçamento em modo matemático}
\begin{block}{Espaçamento matemático}
\begin{center}
\begin{tabular}{ll}
\LCmd{,} 		& espaço pequeno \\
\LCmd{quad} 	& espaço grande \\
\LCmd{qquad} 	& espaço maior
\end{tabular}
\end{center}
\end{block}
\end{frame}

\begin{frame}{Uso do espaçamento}
\begin{LaTeXcode}[Use \LCmd{quad} para separar expressões diferentes]
\LCmd{[}e\string^\Larg{-\LCmd{alpha} t} \LCmd{quad} x\string_1, x\string_2, x\string_3, \LCmd{ldots} \LCmd{quad} x\string_1+x\string_2+x\string_3+\LCmd{cdots}
\end{LaTeXcode}
Produz:
\begin{LaTeXoutput}
\[e^{-\alpha t} \quad x_1, x_2, x_3, \dots \quad x_1+x_2+x_3+\cdots
\]
\end{LaTeXoutput}

\begin{LaTeXcode}[Use \LCmd{qquad} para separar mais as expressões]
\LCmd{[}e\string^\Larg{-\LCmd{alpha} t} \LCmd{qquad} x\string_1, x\string_2, x\string_3, \LCmd{ldots} \LCmd{qquad} x\string_1+x\string_2+x\string_3+\LCmd{cdots}
\end{LaTeXcode}
Produz:
\begin{LaTeXoutput}
\[e^{-\alpha t} \qquad x_1, x_2, x_3, \dots \qquad x_1+x_2+x_3+\cdots
\]
\end{LaTeXoutput}
\end{frame}

\begin{frame}{Uso do espaçamento}
\begin{block}{Observações}
\begin{itemize}
\item Observe o uso de \texttt{\string\ldots} e \texttt{\string\cdots};
\item Esses comandos poderiam ter sido substituidos por \texttt{\string\dots} que funciona bem sempre.
\end{itemize}
\end{block}
\end{frame}

\begin{frame}{Uso do espaçamento \LCmd{,}}
Use \LCmd{,} para colocar ponto final em fórmula:
\begin{LaTeXcode}[Exemplo de uso do \LCmd{,}]
A simplificação desta expressão
resulta em\n
\texttt{\LCmd{[}\LCmdArg{frac}{(x+1)(x-1)}\Larg{y-1}\LCmd{,}.\LCmd{]}}
\end{LaTeXcode}
que produz:
\begin{LaTeXoutput}
A simplificação desta expressão resulta em
\[\frac{(x+1)(x-1)}{y-1}\, .\]
\end{LaTeXoutput}
\end{frame}

\begin{frame}{Uso do espaçamento \LCmd{,}}
Use \LCmd{,} para separar os diferenciais da expressão integranda nos integrais:
\begin{LaTeXcode}
\LCmd{[}
\LCmd{iint}\LCmd{exp}(x\string^2 + y\string^2)\LCmd{,}\LCmdArg{mathrm}{d}x\LCmd{,}\LCmdArg{mathrm}{d}y
\LCmd{]}
\end{LaTeXcode}
Produz:
\begin{LaTeXoutput}
\[
\iint\exp (x^2 + y^2) \,\mathrm{d}x \,\mathrm{d}y
\]
\end{LaTeXoutput}
\end{frame}

\begin{frame}{Raízes}
\begin{LaTeXcode}[Raiz quadrada]
\LCmdArg{sqrt}{x+1}
\end{LaTeXcode}
Produz:
\begin{LaTeXoutput}
\[\sqrt{x+1}\]
\end{LaTeXoutput}
e
\begin{LaTeXcode}[Raiz $n$-ésima]
\LOA sqrt[3]{2}
\end{LaTeXcode}
Produz:
\begin{LaTeXoutput}
\[\sqrt[3]{2}\]
\end{LaTeXoutput}
\end{frame}

\begin{frame}{\LCmd{overline}, \LCmd{underline}, \LCmd{overbrace} e \LCmd{underbrace}}


\begin{LaTeXcode}[\LCmd{overline}]
\LCmdArg{overline}{a+b}
\end{LaTeXcode}
Produz:
\begin{LaTeXoutput}
\[\overline{a+b}\]
\end{LaTeXoutput}
e
\begin{LaTeXcode}[\LCmd{underbrace}]
10110\LCmdArg{underbrace}{111\LCmd{dots}1}\string_\Larg{\string\times\ 56}000
\end{LaTeXcode}
Produz:
\begin{LaTeXoutput}
\[10110\underbrace{111\dots1}_{\times 56}000\]
\end{LaTeXoutput}
\end{frame}

\begin{frame}{Derivada}
\begin{LaTeXcode}[Derivadas]
y=x\string^2 \LCmd{qquad} y'=2x \LCmd{qquad} y'{}'=2
\end{LaTeXcode}
Produz:
\begin{LaTeXoutput}
\[y=x^2 \qquad y'=2x \qquad y''=2\]
\end{LaTeXoutput}

\begin{LaTeXcode}[Derivadas como frações]
y=x\string^2 \LCmd{qquad} \LCmdArg{frac}{\string\mathrm\{d\}y}\Larg{\string\mathrm\{d\}x}=2x \LCmd{qquad} \LCmdArg{frac}{\string\mathrm\{d\}\string^2y}\Larg{\string\mathrm\{d\}x\string^2}=2
\end{LaTeXcode}
Produz:
\begin{LaTeXoutput}
\[
y=x^2 \qquad \frac{\mathrm{d}y}{\mathrm{d}x} =2x \qquad \frac{\mathrm{d}^2y}{\mathrm{d}x^2} =2
\]
\end{LaTeXoutput}

\end{frame}

\begin{frame}{Vetores}
 Use \LCmd{vec}, \LCmd{overrightarrow}, e \LCmd{overleftarrow}. 
\begin{LaTeXcode}[Exemplo]
\LCmd{vec} a \qquad \LCmdArg{overrightarrow}{AB} \qquad
\LCmdArg{overleftarrow}{AB}
\end{LaTeXcode}
Produz:
\begin{LaTeXoutput}
\[\vec a \qquad \overrightarrow{AB} \qquad \overleftarrow{AB}\]
\end{LaTeXoutput}
\end{frame}

\begin{frame}{Coeficientes binomiais}
Use o pacote \Lsty{amsmath}.

\begin{LaTeXcode}
\LCmdArg{binom}{n}\Larg{k} = \n
\null\qquad\LCmdArg{frac}{(n)(n-1)\LCmd{cdots}(n-k+1)}\Larg{(1)(2)\LCmd{cdots}(k)}
\end{LaTeXcode}
Produz:
\begin{LaTeXoutput}
\[\binom{n}{k} = \frac{(n)(n-1)\cdots(n-k+1)}{(1)(2)\cdots(k)}\]
\end{LaTeXoutput}
\end{frame}

\begin{frame}{Delimitadores}
\fontsize{10}{11}\selectfont

Usa-se \LCmd{left} e \LCmd{right} para determinar automaticamente o tamanho dos delimitadores esquerdo e direito. Usa-se \LCmd{bigl}, \LCmd{Bigl}, \LCmd{biggl}, \LCmd{Biggl} e \LCmd{bigr}, \LCmd{Bigr}, \LCmd{biggr}, \LCmd{Biggr} para fixar determinados tamanhos dos delimitadores esquerdo e direito.

\begin{LaTeXcode}[Exemplo]
x+\LCmd{left}(\LCmdArg{frac}{1}\Larg{x+1}\LCmd{right})\string^3
\end{LaTeXcode}
Produz:
\begin{LaTeXoutput}
\[x+\left(\frac{1}{x+1}\right)^3\]
\end{LaTeXoutput}

\begin{LaTeXcode}[Outro exemplo]
\LCmd{Bigl}((x+1)(x-1)\LCmd{Bigr})\string^2
\end{LaTeXcode}
Produz:
\begin{LaTeXoutput}
\[\Bigl((x+1)(x-1)\Bigr)^2\]
\end{LaTeXoutput}
\end{frame}

\begin{frame}{Delimitadores de tamanho determinado}
\begin{itemize}
\item Os descritores de tamanho podem ser usados com qualquer delimitador.

\begin{LaTeXcode}
\LCmd{bigl}(\LCmd{Bigl}(\LCmd{biggl}(\LCmd{Biggl}(\n
\null\qquad\LCmd{bigr}\LCmd{\}}\LCmd{Bigr}\LCmd{\}}\LCmd{biggr}\LCmd{\}}\LCmd{Biggr}\LCmd{\}}
\end{LaTeXcode}
Produz:
\begin{LaTeXoutput}
\[\big(\Big(\bigg(\Bigg(\quad\big\}\Big\}\bigg\}\Bigg\}\]
\end{LaTeXoutput}

\item As terminações  \texttt{l} (\emph{left}, esquerda) e de \texttt{r} (\emph{r}, direita) determina os espaços corretos quando o delimitador é de esquerda ou de direita.
\end{itemize}
\end{frame}

\begin{frame}{\Lenv{eqnarray} e \Lenv{align}}
\begin{itemize}
\item Ambiente \Lenv{eqnarray} foi desenvolvido para mostrar listas de fórmulas como tabelas de três colunas alinhadas na coluna do meio (onde normalmente está o ``='' );
\item Ambiente \Lenv{eqnarray} está obsoleto, pois foi o primeiro ambiente desenvolvido para o \LaTeX{} e possui um erro de espaçamento;
\item Preferível usar o ambiente \Lenv{align}, carregando o pacote \Lsty{amsmath};
\item Assim como existe o ambiente \Lenv{eqnarray*}, também existe o ambiente  \Lenv{align*} nos quais  as equações não são numeradas.
\end{itemize}
\end{frame}

\begin{frame}{Exemplos de uso do ambiente \Lenv{align}}
\begin{LaTeXcode}[Primeiro exemplo]
\LCmdArg{begin}{align}\n
f(x) \& =  x\string^2 \LCmd{\bs}\n
f'(x) \& =  2x \LCmd{\bs}\n
\LCmd{int}\string_0\string^x f(y)\LCmd{,}\LCmdArg{mathrm}{d}y \& = \LCmdArg{frac}{x\string^3}\Larg{3}
\LCmdArg{end}{align}
\end{LaTeXcode}
Produz:
\begin{LaTeXoutput}
\begin{align}
f(x) & = x^2 \\
f'(x) & = 2x \\
\int_0^x f(y)\,\mathrm{d}y & =  \frac{x^3}{3}
\end{align}
\end{LaTeXoutput}

\end{frame}

\begin{frame}{Exemplos de uso do ambiente \Lenv{align}}
\begin{LaTeXcode}[Segundo exemplo]
\LCmdArg{begin}{align}
\LCmd{sin} x \& =  x -\LCmdArg{frac}{x\string^3}\Larg{3!}+
\LCmdArg{frac}{x\string^5}\Larg{5!}- \LCmd{notag} \LCmd{\bs}\n
\&\LCmd{qquad} \LCmdArg{frac}{x\string^7}\Larg{7!}+\LCmd{cdots}
\LCmdArg{end}{align}
\end{LaTeXcode}
Produz:
\begin{LaTeXoutput}
\begin{align}
\sin x & =  x -\frac{x^3}{3!}+
\frac{x^5}{5!}- \notag \\
&\qquad  \frac{x^7}{7!}+\cdots
\end{align}
\end{LaTeXoutput}
\begin{block}{Observação}
\texttt{\string\notag} elimina a numeração na linha.
\end{block}
\end{frame}

\begin{frame}{Descrevendo variáveis}
\begin{LaTeXcode}[Descrição das variáveis]
\LCmd{[}a\string^2+b\string^2=c\string^2\LCmd{]}
\nn
\LCmdArg{begin}{tabular}\Larg{lp\lb.8\string\textwidth\rb}\n
Onde: \& \$a\$, \$b\$ -{}- são os catetos
de um triângulo retângulo\LCmd{tabularnewline}\n
\& \$c\$ -{}- é a hipotenusa do triângulo retângulo.
\LCmdArg{end}{tabular}
\end{LaTeXcode}

Produz:
\begin{LaTeXoutput}
\[a^2+b^2=c^2\]

\begin{tabular}{lp{.8\textwidth}}
Onde: & $a$, $b$ -- são os catetos
de um triângulo retângulo\tabularnewline
& $c$ -- é a hipotenusa do triângulo retângulo.
\end{tabular}
\end{LaTeXoutput}
\end{frame}

\begin{frame}{Descrevendo variáveis}
\begin{LaTeXcode}[Usando \texttt{\string\parindent}]
\LCmd{[}a\string^2+b\string^2=c\string^2\LCmd{]}
\nn
\{\LCmdArg{settowidth}{\string\parindent}\{Onde:\string\ \}\n
\string\noindent\ Onde:\string\ \$a\$, \$b\$ -{}- são os catetos
de um triângulo retângulo\n\n
\$c\$ -{}- é a hipotenusa do triângulo retângulo.\}
\end{LaTeXcode}

Produz:
\begin{LaTeXoutput}
\[a^2+b^2=c^2\]

{\settowidth{\parindent}{Onde:\ }
\noindent Onde: $a$, $b$ -- são os catetos
de um triângulo retângulo

$c$ -- é a hipotenusa do triângulo retângulo.}
\end{LaTeXoutput}
\end{frame}

\begin{frame}{Símbolos matemáticos}

\begin{center}\tabcolsep=4pt
\begin{tabular}{*8l}
\toprule
\multicolumn8c{\bfseries Letras gregas}\\
\midrule
$\alpha$      	& \LCmd{alpha} 		&
$\beta$       	& \LCmd{beta} 		&
$\gamma$      	& \LCmd{gamma} 		&
$\delta$      	& \LCmd{delta} 		\\
$\epsilon$    	& \LCmd{epsilon} 	&
$\varepsilon$	& \LCmd{varepsilon}&
$\zeta$			& \LCmd{zeta} 		&
$\eta$			& \LCmd{eta} 		\\
$\theta$		& \LCmd{theta} 		&
$\vartheta$		& \LCmd{vartheta} 	&
$\iota$			& \LCmd{iota} 		&
$\kappa$		& \LCmd{kappa} 		\\
$\lambda$		& \LCmd{lambda} 	&
$\mu$			& \LCmd{mu} 		&
$\nu$			& \LCmd{nu} 		&
$\xi$			& \LCmd{xi} 		\\
$\pi$			& \LCmd{pi} 		&
$\varpi$		& \LCmd{varpi} 		&
$\rho$			& \LCmd{rho} 		&
$\varrho$		& \LCmd{varrho} 	\\
$\sigma$		& \LCmd{sigma} 		&
$\varsigma$		& \LCmd{varsigma} 	&
$\tau$			& \LCmd{tau} 		&
$\upsilon$  	& \LCmd{upsilon} 	\\
$\phi$			& \LCmd{phi} 		&
$\varphi$		& \LCmd{varphi} 	&
$\chi$			& \LCmd{chi} 		&
$\psi$			& \LCmd{psi} 		\\
$\omega$		& \LCmd{omega}		&
$\Gamma$		& \LCmd{Gamma} 		&
$\Delta$		& \LCmd{Delta} 		&
$\Theta$		& \LCmd{Theta} 		\\
$\Lambda$		& \LCmd{Lambda} 	&
$\Xi$			& \LCmd{Xi} 		&
$\Pi$			& \LCmd{Pi} 		&
$\Sigma$		& \LCmd{Sigma} 		\\
$\Upsilon$ 		& \LCmd{Upsilon} 	&
$\Phi$			& \LCmd{Phi} 		&
$\Psi$			& \LCmd{Psi} 		&
$\Omega$		& \LCmd{Omega} 		\\
\bottomrule
\end{tabular}
\end{center}
\end{frame}

\begin{frame}{Operações binárias}
\fontsize{10}{11}\selectfont
\begin{center}\tabcolsep=3pt
\begin{tabular}{*8l}
\toprule
\multicolumn8c{\bfseries Operações binárias}\\
\midrule
$\pm$		& \LCmd{pm} &
$\mp$		& \LCmd{mp} &
$\times$	& \LCmd{times} &
$\div$		& \LCmd{div} \\
$\ast$		& \LCmd{ast} &
$\star$		& \LCmd{star} &
$\circ$		& \LCmd{circ} &
$\bullet$	& \LCmd{bullet} \\
$\cap$		& \LCmd{cap} &
$\cup$      & \LCmd{cup} &
$\uplus$	& \LCmd{uplus} &
$\sqcap$	& \LCmd{sqcap} \\
$\sqcup$	& \LCmd{sqcup} &
$\vee$		& \LCmd{vee} &
$\wedge$	& \LCmd{wedge} &
$\setminus$	& \LCmd{setminus}\\
$\bigtriangleup$ & \LCmd{bigtriangleup} &
$\cdot$		& \LCmd{cdot} &
$\diamond$	& \LCmd{diamond} &
$\wr$		& \LCmd{wr} \\
$\bigtriangledown$ & \LCmd{bigtriangledown} &
$\lhd$		& \LCmd{lhd} &
$\rhd$		& \LCmd{rhd} &
$\amalg$	& \LCmd{amalg} \\
$\triangleleft$ & \LCmd{triangleleft} &
$\bigcirc$	& \LCmd{bigcirc} &
$\unrhd$	& \LCmd{unrhd} &
$\unlhd$	& \LCmd{unlhd} \\
$\triangleright$ & \LCmd{triangleright} &
$\oplus$	& \LCmd{oplus} &
$\ominus$	& \LCmd{ominus} &
$\otimes$	& \LCmd{otimes} \\
$\oslash$	& \LCmd{oslash} &
$\odot$		& \LCmd{odot} &
$\dagger$ 	& \LCmd{dagger} &
$\ddagger$	& \LCmd{ddagger} \\
\bottomrule
\end{tabular}
\end{center}
\end{frame}

\begin{frame}{Relações binárias}
\begin{center}%\tabcolsep3pt
\begin{tabular}{*6l}
\toprule
\multicolumn6c{\bfseries Relações binárias}\\
\midrule
$\leq$		& \LCmd{leq} &
$\prec$		& \LCmd{prec} &
$\preceq$	& \LCmd{preceq} \\
$\ll$		& \LCmd{ll} &
$\subset$	& \LCmd{subset} &
$\subseteq$	& \LCmd{subseteq} \\
$\sqsubset$	& \LCmd{sqsubset} &
$\sqsubseteq$   & \LCmd{sqsubseteq} &
$\in$		& \LCmd{in} \\
$\ni$		& \LCmd{ni} &
$\dashv$	& \LCmd{dashv} &
$\equiv$	& \LCmd{equiv} \\
$\sim$		& \LCmd{sim} &
$\simeq$	& \LCmd{simeq} &
$\asymp$	& \LCmd{asymp} \\
$\approx$	& \LCmd{approx} &
$\cong$		& \LCmd{cong} &
$\neq$		& \LCmd{neq} \\
$\vdash$	& \LCmd{vdash} &
$\geq$		& \LCmd{geq} &
$\succ$		& \LCmd{succ} \\
$\succeq$	& \LCmd{succeq} &
$\gg$		& \LCmd{gg} &
$\supset$	& \LCmd{supset} \\
$\supseteq$	& \LCmd{supseteq} &
$\sqsupset$	& \LCmd{sqsupset} &
$\sqsupseteq$	& \LCmd{sqsupseteq}\\
$\doteq$	& \LCmd{doteq} &
$\propto$	& \LCmd{propto} &
$\models$	& \LCmd{models} \\
$\perp$		& \LCmd{perp} &
$\mid$		& \LCmd{mid} &
$\parallel$	& \LCmd{parallel}\\
$\bowtie$	& \LCmd{bowtie} &
$\Join$		& \LCmd{Join} &
$\smile$	& \LCmd{smile} \\
$\frown$	& \LCmd{frown}\\
\bottomrule
\end{tabular}
\end{center}
\end{frame}

\begin{frame}{Setas}
\fontsize{10}{11}\selectfont

\begin{center}\makebox[\textwidth]{\tabcolsep=3pt
\begin{tabular}{*4l}
\toprule
\multicolumn4c{\bfseries Setas}\\
\midrule
$\leftarrow$		& \LCmd{leftarrow} &
$\Leftarrow$		& \LCmd{Leftarrow} \\
$\rightarrow$		& \LCmd{rightarrow} &
$\Rightarrow$		& \LCmd{Rightarrow}\\
$\leftrightarrow$	& \LCmd{leftrightarrow} &
$\Leftrightarrow$ 	& \LCmd{Leftrightarrow}\\
$\mapsto$		& \LCmd{mapsto} &
$\hookleftarrow$	& \LCmd{hookleftarrow} \\
$\leftharpoonup$	& \LCmd{leftharpoonup} &
$\Longleftrightarrow$	& \LCmd{Longleftrightarrow} \\
$\longmapsto$		& \LCmd{longmapsto} &
$\hookrightarrow$	& \LCmd{hookrightarrow} \\
$\rightharpoonup$	& \LCmd{rightharpoonup} &
$\rightharpoondown$	& \LCmd{rightharpoondown} \\
$\leadsto$		& \LCmd{leadsto} &
$\uparrow$		& \LCmd{uparrow} \\
$\Uparrow$		& \LCmd{Uparrow} &
$\downarrow$		& \LCmd{downarrow} \\
$\leftharpoondown$	& \LCmd{leftharpoondown} &
$\rightleftharpoons$	& \LCmd{rightleftharpoons} \\
$\longleftarrow$	& \LCmd{longleftarrow} &
$\Longleftarrow$	& \LCmd{Longleftarrow} \\
$\longrightarrow$	& \LCmd{longrightarrow} &
$\Longrightarrow$       & \LCmd{Longrightarrow} \\
$\longleftrightarrow$	& \LCmd{longleftrightarrow} &
$\Downarrow$		& \LCmd{Downarrow}\\
$\updownarrow$		& \LCmd{updownarrow} &
$\Updownarrow$		& \LCmd{Updownarrow} \\
$\nearrow$		& \LCmd{nearrow} &
$\searrow$		& \LCmd{searrow} \\
$\swarrow$		& \LCmd{swarrow} &
$\nwarrow$		& \LCmd{nwarrow} \\
\bottomrule
\end{tabular}}
\end{center}
\end{frame}

\begin{frame}{Micelânea}
\begin{center}\tabcolsep=2pt
\begin{tabular}{*8l}
\toprule
\multicolumn8c{\bfseries Micelânea}\\
\midrule
$\aleph$		& \LCmd{aleph} &
$\hbar$			& \LCmd{hbar} &
$\imath$		& \LCmd{imath} &
$\jmath$		& \LCmd{jmath} \\
$\ell$			& \LCmd{ell} &
$\wp$			& \LCmd{wp} &
$\Re$			& \LCmd{Re} &
$\Im$			& \LCmd{Im} \\
$\mho$			& \LCmd{mho} &
$\angle$		& \LCmd{angle} &
$\forall$		& \LCmd{forall} &
$\exists$		& \LCmd{exists} \\
$\neg$			& \LCmd{neg} &
$\flat$			& \LCmd{flat} &
$\natural$		& \LCmd{natural} &
$\sharp$		& \LCmd{sharp} \\
$\backslash$	& \LCmd{backslash} &
$\partial$		& \LCmd{partial} &
$\prime$		& \LCmd{prime} &
$\emptyset$		& \LCmd{emptyset} \\
$\nabla$		& \LCmd{nabla} &
$\surd$			& \LCmd{surd} &
$\top$			& \LCmd{top} &
$\bot$			& \LCmd{bot} \\
$\|$			& \LCmd{|} &
$\Box$			& \LCmd{Box} &
$\Diamond$		& \LCmd{Diamond} &
$\triangle$		& \LCmd{triangle} \\
$\spadesuit$	& \LCmd{spadesuit} &
$\clubsuit$		& \LCmd{clubsuit} &
$\diamondsuit$	& \LCmd{diamondsuit} &
$\heartsuit$	& \LCmd{heartsuit} \\
$\infty$		& \LCmd{infty} \\
\bottomrule
\end{tabular}
\end{center}
\end{frame}

\begin{frame}{Símbolos de tamanho variável}
\begin{center}
\begin{tabular}{*4l}
\toprule
\multicolumn4c{\bfseries Símbolos de tamanho variável}\\
\midrule
$\sum$		& \LCmd{sum} &
$\prod$		& \LCmd{prod} \\
$\coprod$	& \LCmd{coprod} &
$\int$		& \LCmd{int} \\
$\oint$		& \LCmd{oint} &
$\bigcap$	& \LCmd{bigcap} \\
$\bigcup$	& \LCmd{bigcup} &
$\bigsqcup$	& \LCmd{bigsqcup} \\
$\bigvee$	& \LCmd{bigvee} &
$\bigwedge$	& \LCmd{bigwedge} \\
$\bigodot$	& \LCmd{bigodot} &
$\bigotimes$	& \LCmd{bigotimes} \\
$\bigoplus$	& \LCmd{bigoplus} &
$\biguplus$	& \LCmd{biguplus} \\
\bottomrule
\end{tabular}
\end{center}
\end{frame}

\begin{frame}{Funções matemáticas}
\begin{LaTeXcode}[Funções matemáticas]
\LCmd{arccos} \LCmd{arcsin} \LCmd{arctan} \LCmd{arg} \LCmd{cos}
\LCmd{cosh} \LCmd{cot} \LCmd{coth} \LCmd{csc} \LCmd{deg} \LCmd{det}
\LCmd{dim} \LCmd{exp} \LCmd{gcd} \LCmd{hom} \LCmd{inf} \LCmd{ker} \LCmd{lg}
\LCmd{lim} \LCmd{liminf} \LCmd{limsup} \LCmd{ln} \LCmd{log} \LCmd{max}
\LCmd{min} \LCmd{Pr} \LCmd{sec} \LCmd{sin} \LCmd{sinh} \LCmd{sup} \LCmd{tan}
\LCmd{tanh}
\end{LaTeXcode}
\end{frame}

\begin{frame}{Arrays}
\fontsize{10}{11}\selectfont

O ambiente \Lenv{array} permite descrever material matemático em formato de matriz, com linhas e  colunas.

\begin{LaTeXcode}[Exemplo]
\LCmdArg{begin}{array}\Larg{clcr}\n
a+b+c \& uv    \& x-y \& 27  \LCmd{\bs}\n
a+b   \& u+v   \& z   \& 134 \LCmd{\bs}\n
a     \& 3u+vw \& xyz \& 2,978 \LCmd{\bs}\n
\LCmdArg{end}{array}
\end{LaTeXcode}
Produz:
\begin{LaTeXoutput}
\[
\begin{array}{clcr}
a+b+c & uv    & x-y & 27  \\
a+b   & u+v   & z   & 134 \\
a     & 3u+vw & xyz & 2{,}978 \\
\end{array}
\]
\end{LaTeXoutput}

\begin{block}{Observação}
Os descritores de colunas \texttt{clcr} são somente para exemplificar; normalmente as colunas das matrizes tem seu conteúdo centrado.
\end{block}
\end{frame}

\begin{frame}{Matrizes delimitadas}
Matrizes podem ser obtidas usando-se delimitadores (``\lb'', ``\ls'', 
``(''). Para indicar se o delimitador é o esquerdo ou o direito 
anteceder o delimitador por \LCmd{left} ou \LCmd{right}.

\begin{LaTeXcode}[Exemplo]
\LCmd{[} \LCmd{left}\ls\ 
\LCmdArg{begin}{array}\Larg{*4c}\n
a+b+c \& uv    \& x-y \& 27  \LCmd{\bs}\n
a+b   \& u+v   \& z   \& 134 \LCmd{\bs}\n
a     \& 3u+vw \& xyz \& 2,978 \LCmd{\bs}\n
\LCmdArg{end}{array}
\LCmd{right}\rs\ \LCmd{]}
\end{LaTeXcode}

Produz:
\begin{LaTeXoutput}
\[ \left [
\begin{array}{*4c}
a+b+c & uv    & x-y & 27  \\
a+b   & u+v   & z   & 134 \\
a     & 3u+vw & xyz & 2{,}978 \\
\end{array}
\right ] \]
\end{LaTeXoutput}
\end{frame}

\begin{frame}{Matrizes}
Mais um exemplo:
\begin{LaTeXcode}[Usando ``('' como delimitador]
\LCmd{[} \LCmd{left}(
\LCmdArg{begin}{array}\Larg{*3c}\n
a\string_\{11\} \& a\string_\{12\} \& \LCmd{dots} \LCmd{\bs}\n
a\string_\{21\} \& a\string_\{22\} \& \LCmd{dots} \LCmd{\bs}\n
\LCmd{vdots}  \& \LCmd{vdots}  \& \LCmd{ddots}          \n
\LCmdArg{end}{array} \LCmd{right}) \LCmd{]}
\end{LaTeXcode}
Produz:
\begin{LaTeXoutput}
\[ \left (
\begin{array}{ccc}
a_{11} & a_{12} & \ldots \\
a_{21} & a_{22} & \ldots \\
\vdots & \vdots & \ddots
\end{array} \right ) \]
\end{LaTeXoutput}
\end{frame}

\begin{frame}{Delimitador vazio}
\begin{itemize}
\item O delimitador vazio produz-se com um ponto: \LCmd{right.}
\item Serve para mostrar opções usando chaves

\begin{LaTeXcode}[Exemplo]
f(x)=\LCmd{left}\LCmd{\lb}\n
\LCmdArg{begin}{array}\Larg{ll}\n
0 \& x\LCmd{leq} 0 \LCmd{\bs}\n
x\string^2 \& x>0\n
\LCmdArg{end}{array}\n\LCmd{right}.
\end{LaTeXcode}
Produz:
\begin{LaTeXoutput}
\[
f(x)=\left\{
\begin{array}{ll}
0 & x\leq 0 \\
x^2 & x>0
\end{array}\right.
\]
\end{LaTeXoutput}

\item O pacote \Lsty{amsmath} oferece o ambiente \Lenv{cases} que permite obter mas diretamente o mesmo resultado.
\end{itemize}
\end{frame}
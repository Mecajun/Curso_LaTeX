%% Aula 5 - Beamer

\begin{frame}{Produzindo apresentações com Seminar}
\begin{itemize}
\item \Lsty{Seminar} é incluído no \TeX\ Live
\begin{LaTeXcode}[Declaração]
\LOA documentclass[slideonly,12pt]{seminar}
\end{LaTeXcode}
\item Para obter frame e sombreamento:
\begin{LaTeXcode}[Frame e sombreamento]
\LCmdArg{usepackage}{fancybox}\n
\LCmdArg{slideframe}{shadow}
\end{LaTeXcode}
\end{itemize}
\end{frame}

\begin{frame}{Seminar}

\begin{itemize}
\item Para definir um slide:
\begin{LaTeXcode}[Slide]
\LCmdArg{begin}{slide}\n
\dots\n
\LCmdArg{end}{slide}
\end{LaTeXcode}
\item Para continuar nos slides seguintes:
\begin{LaTeXcode}[Quebra de slide]
\LCmd{newslide}
\end{LaTeXcode}
\end{itemize}
\end{frame}

\begin{frame}{Beamer}
\begin{itemize}
\item Apresentações mais dinâmicas;
\item Incluído no \TeX\ Live;
\item Requer também os pacotes \Lsty{pgf} e \Lsty{xcolor};
\item Veja: \url{http://minerva.ufpel.edu.br/~campani/tutbeamer.tar.gz}
\item Uso:
\begin{itemize}
\item \LCmdArg{documentclass}{beamer};
\item Estrutura usando \LCmd{section} e \LCmd{subsection};
\item Slides individuais dentro de comandos \LCmd{frame};
\item Compilar direitamente com \prog{pdflatex}.
\item Veja: \url{http://www.tug.org/teTeX/tetex-texmfdist/doc/latex/beamer/beameruserguide.pdf}
\end{itemize}
\end{itemize}

\end{frame}

\begin{frame}{Exemplo de documento \LCmd[]{beamer} }
\fontsize{10}{11}\selectfont

\begin{LaTeXcode}[Exemplo]
\LCmdArg{documentclass}{beamer} \n
\LCmdArg{usepackage}{beamerthemesplit} \n
\LCmdArg{title}{Exemplo} \n
\LCmdArg{author}{Till Tantau} \n
\LCmdArg{begin}{document} \n
\LCmdArg{frame}{\LCmd{titlepage}} \n
\LOA section[Outline]{} \n
\LCmdArg{frame}{\LCmd{tableofcontents}} \n
\LCmdArg{section}{Introdução} \n
\LCmdArg{subsection}{Visão geral da classe Beamer} \n
\LCmdArg{begin}{frame}\Larg{Características da classe Beamer} \n
\null\quad  \LCmdArg{begin}{itemize} \n
\null\qquad  \LCmd{item}<1-> Classe \LCmd{LaTeX}\LCmd{ }normal. \n
\null\qquad  \LCmd{item}<2-> Fácil sobreposição. \n
\null\qquad  \LCmd{item}<3-> Sem necessidade de programas externos.  \n     
\null\quad  \LCmdArg{end}{itemize} \n
\LCmdArg{end}{frame} \n
\LCmdArg{end}{document} \n
\end{LaTeXcode}
\end{frame}

\begin{frame}{Alguns comandos de \LCmd[]{beamer}}
\begin{LaTeXcode}[Temas]
\LCmdArg{usetheme}{\dots}
\end{LaTeXcode}

\begin{LaTeXcode}[Frames]
\LCmdArg{begin}{frame}\Larg{Título do frame} \n
\dots \n
\LCmdArg{end}{frame} \nn
ou \nn
\LCmdArg{frame}{\LCmdArg{frametitle}{Título do frame} \n
\dots \n
}
\end{LaTeXcode}
\end{frame}

\begin{frame}{Alguns comandos de \LCmd[]{beamer}}
\begin{LaTeXcode}[Logo]
\LOA pgfdeclareimage[height=1.4cm]{logo}\Larg{unb}
\LCmdArg{logo}{\LCmdArg{pgfuseimage}{logo}}
\end{LaTeXcode}

\begin{block}{Observação}
arquivo de imagem: \texttt{unb.jpg} (retira-se a extensão)
\end{block}

\begin{LaTeXcode}[Blocos]
\LCmdArg{begin}{block}\Larg{Título do bloco} \n
\dots \n
\LCmdArg{end}{block}
\end{LaTeXcode}
\end{frame}

\begin{frame}{Colunas}
\begin{LaTeXcode}[Colunas]
\LCmdArg{begin}{columns}\LO[t]
 \nn
\LCmdArg{begin}{column}\Larg{5cm} \n
\dots \n
\LCmdArg{end}{column}
 \nn
\LCmdArg{begin}{column}\Larg{5cm} \n
\dots \n
\LCmdArg{end}{column}
 \nn
\LCmdArg{end}{columns}
\end{LaTeXcode}
\end{frame}

\begin{frame}{Overlays}
	\begin{LaTeXcode}[Overlays]
		\LCmdArg{begin}{itemize} \n
		\LCmd{item} <1-> Primeira coisa \n
		\LCmd{item} <2-> Segunda coisa \n
		\LCmd{item} <3-> Terceira coisa \n
		\LCmdArg{end}{itemize} \n
	\end{LaTeXcode}
	
	\begin{itemize}
	\item Especificação de overlay:
	\begin{itemize}
		\item \texttt{<3->} -- mostra do 3 em diante;
		\item \texttt{<2-5>} -- mostra entre o 2 e o 5;
		\item \texttt{<-4>} -- mostra até o 4.
	\end{itemize}
\end{itemize}
\end{frame}

\begin{frame}{Transparência}
Para obter transparência: 

\LCmdArg{setbeamercovered}{transparent}
  e usar  \LCmd{uncover} em substituição aos \LCmd{item}.
\end{frame}

\begin{frame}{Destacando}
\begin{LaTeXcode}[Destacando]
\LCmdArg{begin}{itemize} \n
\LCmd{item} <1- | alert@1> Primeira coisa \n
\LCmd{item} <2- | alert@2> Segunda coisa \n
\LCmd{item} <3- | alert@3> Terceira coisa \n
\LCmdArg{end}{itemize}
\end{LaTeXcode}
\end{frame}

\begin{frame}{Overlays com blocos}
\begin{LaTeXcode}[Overlays com blocos]
\LCmdArg{begin}{frame}\Larg{Overlays com blocos} \n
\LCmdArg{begin}{block}\Larg{Primeiro bloco}<1-> \n
Este é o primeiro bloco \n
\LCmdArg{end}{block}
\nn
\LCmdArg{begin}{block}\Larg{Segundo bloco}<2-> \n
Este é o segundo bloco \n
\LCmdArg{end}{block} \n
\LCmdArg{end}{frame}
\end{LaTeXcode}
\end{frame}

\begin{frame}{Efeitos nas transições de lâminas}

\begin{itemize}
\item \LCmd{transdissolve}
\item \LCmd{transsplitverticalout}
\item \LCmd{transblindshorizontal}
\item etc.
\end{itemize}
\end{frame}

\documentclass{article}
\usepackage[utf8]{inputenc}
\usepackage{minted}
\usepackage[brazilian]{babel}

% use lualatex to compile
 
%% \luaTable{'numberOfColumns'}{'dataFile.dat'}{'legend'}
%% {'First Line of table (title)'}
 
\usepackage{luacode} %% Package to use the enviroment luacode
 
\begin{luacode}
numC = 2
 
function readfileDat(filename)
    local filename = "./data/"..filename
 
    for line in io.lines(filename) do
        local numl = {}
        for n in string.gmatch(line,"[\%d\%.]+") do
            numl[#numl+1] = tostring(n)
        end

        numC = #numl
        tex.sprint(table.concat(numl," & "),"\\\\")
    end
end
 
function readfileDat0(filename)
    local tableData = {}
    tableData,numC = readfileDat0(filename)
    tex.sprint(table.concat(tableData,"\\\\"))
end
 
\end{luacode}
 
\newcommand{\luaTable}[5][\directlua{tex.print(numC)}]
{
\begin{table}[H]
\centering
\label{#5}
\caption{#3}
\begin{tabular}{*{#1}{c}}
\hline
#4\\
\hline
\directlua{readfileDat('#2')}
\hline
\end{tabular}
\end{table}
}
 
% Sample:
% \luaTable[3]{teste.dat}{legenda}{x&y&z}
 
% -----------------------------------------------------------------
 
\newcounter{cGraph}
\setcounter{cGraph}{1}
\newcommand{\plotedFigure}[2][\thecGraph]{
\begin{figure}[H]
\centering
\label{fig:g#1}
\includegraphics[width=10cm]{../image/graph#1.png}
\caption{#2}
\end{figure}
\stepcounter{cGraph}
}
 
% Sample:
% \plotedFigure{numberOfGraph}{'legend'}
% -----------------------------------------------------------------

\title{Dummie-Lua}
\author{Rafael Lima}
\date{February 2014}

\begin{document}

\usemintedstyle{tango}

\maketitle

\section{Gerando uma tabela a partir de um arquivo de texto}
\begin{minted}{lua}
numC = 2
 
function readfileDat(filename)
    local filename = "./data/"..filename
 
    for line in io.lines(filename) do
        local numl = {}
        for n in string.gmatch(line,"[\%d\%.]+") do
            numl[#numl+1] = tostring(n)
        end

        numC = #numl
        tex.sprint(table.concat(numl," & "),"\\\\")
    end
end
\end{minted}


\luaTable[2]{table1.dat}{legenda}{x&y}{}

\section{Metodo de Crammer}

\newcommand{\printLua}[1]{\directlua{tex.print(#1)}}

\begin{luacode}
-- Definindo faixa de valores:
bias = 20
range = 50 

-- Escrevendo uma mensagem:
tex.sprint("Gerando um sistema linear com valores inteiros aleatorios entre ",
   -bias," e ",range-bias,":")

-- Gerando os valores aleatorios:
a11 = math.floor(math.random()*range - bias)
a12 = math.floor(math.random()*range - bias)
a21 = math.floor(math.random()*range - bias)
a22 = math.floor(math.random()*range - bias)

b1 = math.floor(math.random()*range - bias)
b2 = math.floor(math.random()*range - bias)

-- Calculando o valor de X e Y no sistema:
D = a11*a22-a12*a21
Dx = b1*a22-b2*a21
Dy = a11*b1-a12*b2

y=Dy/D
x=Dx/D

-- tex.sprint("$$\\begin{array}{l}x = ",x,"\\\\y = ",y,"\\end{array}$$")
\end{luacode}

\begin{equation}
\left\{
\begin{array}{l}
\printLua{a11}\cdot x + \printLua{a12} \cdot y = \printLua{b1}\\
\printLua{a21}\cdot x + \printLua{a22} \cdot y = \printLua{b2}\\
\end{array}
\right.
\end{equation}

\begin{luacode}
tex.sprint("$$\\begin{array}{l}x = ",x,"\\\\y = ",y,"\\end{array}$$")
\end{luacode}

\paragraph{}Código Usado

\begin{minted}{lua}
-- Definindo faixa de valores:
bias = 20
range = 50 

-- Escrevendo uma mensagem:
tex.sprint("Gerando um sistema linear com valores inteiros aleatorios entre ",
    -bias," e ",range-bias,":")

-- Gerando os valores aleatorios:
a11 = math.floor(math.random()*range - bias)
a12 = math.floor(math.random()*range - bias)
a21 = math.floor(math.random()*range - bias)
a22 = math.floor(math.random()*range - bias)

b1 = math.floor(math.random()*range - bias)
b2 = math.floor(math.random()*range - bias)

-- Calculando o valor de X e Y no sistema:
D = a11*a22-a12*a21
Dx = b1*a22-b2*a21
Dy = a11*b1-a12*b2

y=Dy/D
x=Dx/D
\end{minted}



\end{document}

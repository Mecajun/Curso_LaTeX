% use lualatex to compile
 
%% \luaTable{'numberOfColumns'}{'dataFile.dat'}{'legend'}
%% {'First Line of table (title)'}
 
\usepackage{luacode} %% Package to use the enviroment luacode
 
\begin{luacode}
numC = 2
 
function readfileDat(filename)
    local filename = "./data/"..filename
 
    for line in io.lines(filename) do
        local numl = {}
        for n in string.gmatch(line,"[\%d\%.]+") do
            numl[#numl+1] = tostring(n)
        end

        numC = #numl
        tex.sprint(table.concat(numl," & "),"\\\\")
    end
end
 
function readfileDat0(filename)
    local tableData = {}
    tableData,numC = readfileDat0(filename)
    tex.sprint(table.concat(tableData,"\\\\"))
end
 
\end{luacode}
 
\newcommand{\luaTable}[5][\directlua{tex.print(numC)}]
{
\begin{table}[H]
\centering
\label{#5}
\caption{#3}
\begin{tabular}{*{#1}{c}}
\hline
#4\\
\hline
\directlua{readfileDat('#2')}
\hline
\end{tabular}
\end{table}
}
 
% Sample:
% \luaTable[3]{teste.dat}{legenda}{x&y&z}
 
% -----------------------------------------------------------------
 
\newcounter{cGraph}
\setcounter{cGraph}{1}
\newcommand{\plotedFigure}[2][\thecGraph]{
\begin{figure}[H]
\centering
\label{fig:g#1}
\includegraphics[width=10cm]{../image/graph#1.png}
\caption{#2}
\end{figure}
\stepcounter{cGraph}
}
 
% Sample:
% \plotedFigure{numberOfGraph}{'legend'}
% -----------------------------------------------------------------